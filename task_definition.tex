\chapter{Task Description}
\section*{Supervisor}
\begin{itemize}
    \item Prof. Dr. Josef M. Joller, Professor of Computer Science
\end{itemize}

\section*{Problem Description}

The functional programming paradigm becomes more and more integrated within classical
programming languages: Lambdas, streaming API just to name a few. Languages such
as Scala with its rich type system are only understandable if the developer
brings extensive knowledge of a strict functional language with. The past shows
that only highly experienced developer are capable of producing usable and
maintainable Scala code in industry projects.

Due to the supervision, the MAS course as well as the Multi-Paradigm partial
module in the compiler course are based on Haskell.

\section*{Assignment of Tasks}

One goal of this thesis is to work out an overview of the current state of
message-oriented-middleware (MOM), especially approaches of event-streaming.

\par
\begingroup
\leftskip4em
\rightskip\leftskip

Essential outcome of this part is the specification of the in Scala written
message broker "Kafka" which was developed at LinkedIn and stands now under the
Apache license.  Kafka claims to be \textit{fast}, \textit{reliable},
\textit{scalable}, \textit{durable} and \textit{distributed by design}
(documentation of Apache Kafka: \url{http://kafka.apache.org/}).

\par
\endgroup

A further goal is the definition and implementation of the communication
protocol ("wire-protocol") of Kafka using Haskell.

\par
\begingroup
\leftskip4em
\rightskip\leftskip
This part of the thesis is essential since differently (programming language,
operating system) developed clients will communication with Kafka.
\par
\endgroup
The third goal will be the implementation of the basic functionalities of Kafka
in Haskell.

\par
\begingroup
\leftskip4em
\rightskip\leftskip
Since Kafka is strongly decentralized by design, it uses "ZooKeeper" (Apache) on
the underlying layers. This functionality (multi cluster) is beyond the scope of
this bachelors thesis. This thesis will be reduced to only one cluster.
\par
\endgroup
An important goal is performance comparison of the implementation in Haskell
against the one in Scala.

\par
\begingroup
\leftskip4em
\rightskip\leftskip
Additionally Kafka has to be installed as well as a benchmark has to be defined.
\par
\endgroup

\section*{Realization}

In order to work on this assignment, familiarization in the technical basics of
message brokers as well as the programming language Haskell is required.

Solid skills in programming and the will to work within new domains of knowledge
is required as well.

Because of the variable spectrum of this assignment the topic is perfectly
suitable to be handled by two students.

Usually the meetings with the supervisor will be take place on a weekly basis.
Additional meetings will be requested by the students on their needs. Every
meeting has to be planed using an agenda as well as documented using a protocol,
which has to be sent to the supervisor.

To proceed this assignment a project schedule has to be created. Thereby the
focus lies on continuous and transparent work steps. Corresponding the
milestones, early versions of this thesis has to be provided. The students will
then receive feedback. The final grade will be provided based on the version of
the implementation as well as documentation which were delivered at due date.

\section*{Documentation and Delivery}

Since the results are intended to be used in further studies, the priority is set
to the completeness as well as the quality (grammar and graphics) of the
documentation.

The documentation regarding the project planning and tracking has to be
proceeded by the guideline of the computer science department. The detail
requirements regarding the documentation of research as well as the results of
the implementation will be defined according to the concrete task schedule.

The complete documentation has to be handed as three issues of a CD.
Additionally to the documentation the following has to be provided:

\begin{itemize}
    \item a poster to demonstrate the work
    \item all data which is required to understand the results and files (source
      code, build scripts, test code, test data etc.)
\end{itemize}

\section*{Schedule}
\begin{table}[H]
\begin{tabular}{|l|p{12cm}|}
\cline{1-2}
{\bf Spring Semester 2015}     & {\bf Beginning of the thesis. Assignment
description is provided by the supervisor}
  \\ \cline{1-2}
1. Week  & Kick-off Meeting
  \\ \cline{1-2}
2. Week & Delivery of a project plan (draft), including a potential list of
  hardware to be provided
  \\ \cline{1-2}
4. Week  & \begin{tabular}[c]{@{}l@{}} Delivery of early technology research
  report as well as a detailed proposal of an working plan. Fixing of the
project plan together with the supervisor. \end{tabular}   \\
\cline{1-2}
6. Week  & \begin{tabular}[c]{@{}l@{}}Delivery of an implementation proposal
regarding the experimental environment. Scope of functionality and time estimate
coordination and definition together with the supervisor.\end{tabular}              \\
\cline{1-2}
7. Week  & Delivery of requirement and domain analysis for the experimental
environment.
  \\ \cline{1-2}
10. Week & Review meeting regarding software design of experimental environment.
  \\ \cline{1-2}
13. Week & Presentation of the current implementation state.
  \\ \cline{1-2}
Early in June 2015  & \begin{tabular}[c]{@{}l@{}}Delivery of an abstract for the
  thesis booklet as well as the A0-poster to be reviewed by the supervisor.. \end{tabular}                                                     \\
\cline{1-2}
Mid of June 2015   & Delivery of the report as well as the final poster to the
supervisor.
  \\ \cline{1-2}
\end{tabular}
\end{table}
\captionof{table}{Schedule}

The detailed dates will be set by the computer science department.

\section*{References}
\begin{itemize}
    \item [1] Apache Kafka: \url{http://kafka.apache.org}
    \item [2] Apache ZooKeeper: \url{http://zookeeper.apache.org}
    \item [3] Haskell: \\
           Hutton Programming Haskell
           \url{http://www.cs.nott.ac.uk/~gmh/book.html}\\
            Miran Lipovača; Learn you a Haskell
            \url{http://leanyouahaskell.com} \\
            Bryan O'Sullivan, Don Stewart, and John Goerzen  Real World Haskell 
            \url{http://book.realworldhaskell.org}
\end{itemize}

