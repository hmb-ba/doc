\chapter{Personal Reports}

\section*{Report: Marc Juchli}
As messaging never was a main part of my studies, I initially could not
directly imagine what processes are hidden behind a messaging system, nor the
steps it would take to build such a system from scratch.  During the prestudy it
turned out that I let myself into a subject which will soon become one of the
most interesting things I've discovered at HSR. Combined with the functional
programming language Haskell, which I was already confronted with in my semester
abroad, I became even more interested in this thesis. The combination of
constant research and direct implementation was a very interesting challenge.
The fact that we created a working product in the end brought deep satisfaction
to me.

The work with my team partner Lorenz Wolf was more than enjoyable. The endless
discussions as well as constructive pair-programming sessions is part of what
this thesis made a success. I highly honor Lorenz for his patience and efforts
he brought within this thesis.

I confronted the beginning of this thesis with a mix of frustration and
motivation. Messaging as well as Haskell presented themselves as topics with
steep learning curves. It took a vast amount of time to not only get an
understanding but also to being able to produce an outcome.  However, studying
Haskell with the book \textit{Learn you a Haskell for a Great Good} was a great
pleasure and I soon became more and more motivated.  In the end, the gained
knowledge and approaches I learnt during the work with Haskell is what I
treasure most.

I am looking forward to hopefully be able to continue working on this subject
during my master studies.

\newpage
\section*{Report: Lorenz Wolf}

The given task combines two topics I am very interested in: Data integration
with messaging and the concepts of the functional programming language Haskell.
Messaging in general is something I got in touch from time to time during my
studies, but only theoretically. My intent for this bachelor thesis was really to
get in touch with the fundamentals and underlying technologies in detail. To
combine this with intensively learning of the functional programming paradigm was an
additional motivation to choose this challenging topic for my thesis.

The task was predestined for performing in a team of two. The work with my team
partner Marc Juchli arranged very well. We started with nearly the same
knowledge regarding messaging and I could benefit from Marc's first experiences
with Haskell. Constructive discussions and pair programming sessions led to a
very comfortable and efficient work flow.

Although the technology research of this thesis was very interesting and
informative, it was a very hard and time-consuming starting phase. Often it was
quite difficult to look through the \textit{jungle of terminology} which is also
mentioned in the thesis. In such moments it was important to share acquired
knowledge withing the team and motivate each other.

Study Haskell with the book \textit{Learn you a Haskell for a Great Good} was
really a success and a great pleasure too. To actually develop a real world
application with the learned concepts motivated me very much. Furthermore
the topic around Apache Kafka manifested itself as absolutely contemporary issue
these days. Thereby I am certain that our work will get some relevance in
the Haskell as well event streaming community.



