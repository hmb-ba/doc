\chapter{Personal Reports}

\section*{Report: Marc Juchli}

\subsection*{Task}
\subsection*{Teamwork}
\subsection*{Negative Experience}
\subsection*{Positive Experience}


\section*{Report: Lorenz Wolf}
The given task combines two topics I am very interested in: Data integration
with messaging and the concepts of the functional programming language Haskell.
Messaging in general is something I got in touch from time to time during my
studies, but only theoretically. My intent for this bachelor thesis was really to
get in touch with the fundamentals and underlying technologies in detail. To
combine this by intensively learning the functional programming paradigm was an
additional motivation to choose this challenging topic for my thesis.

The task was predestined for performing in a team of two. The work with my team
partner Marc Juchli arranged very well. We started with nearly the same
knowledge regarding messaging and I could benefit from Marc's first experiences
with Haskell. Constructive discussions and pair programming sessions led to a
comfortable and efficient work flow.

Although the technology research of this thesis was very interesting and
informative, it was a very hard and time-consuming starting phase. Often it was
quite difficult to look through the \textit{jungle of terminology} which is also
mentioned in the thesis. In such moments it was important to share acquired
knowledge withing the team and motivate each other.

Study Haskell with the book \textit{Learn you a Haskell for a Great Good} was
really a success and a great pleasure too. To actual develop a real world
application with the learned concepts motivated me very much. Furthermore
the topic around Apache Kafka manifested itself as absolutely contemporary issue
these days. Thereby I am certain that our work will get some relevance in
the Haskell as well event streaming community.



