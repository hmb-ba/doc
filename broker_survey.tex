\chapter{Survey of Message Broker implementation} 
\label{survey-broker}
\todo[inline]{Intro}

\section{Relevant Characteristics}
\begin{description}
    \item [Architecture] \hfill \\
    {}
    \item [Throughput] \hfill \\
        { (What it is) (Key Aspects)}
    \item [Scalability] \hfill \\
    {}
    \item [Latency]\hfill \\
    {}
    \item [Scalability] \hfill \\
    {}
    \item [Reliability / Fault Tolerance] \hfill \\
    {}
    \item [Message delivery] \hfill \\
    {}

\end{description}
\section{Products}
\subsection{Traditional Message Broker}

\begin{description}
    \item [Active MQ] \hfill \\
        {}
    \item [Rabbit MQ] \hfill \\
    {
    Performance / Persistence:
    It is possible for persistence to underperform because the persister is
    limited in the number of file handles or async threads it has to work with.
    In both cases this can happen when you have a large number of queues which
    need to access the disk simultaneously. 
    (https://www.rabbitmq.com/persistence-conf.html)


    In a RabbitMQ Cluster, queues are
    created and live in a single node, and all nodes know about
    all the queues. When a node receives a request to a queue
    that is not available in the current node, it routes the request
    to the node that has the queue.
    To provide high availability
    (HA), RabbitMQ has a mirrored queue arranged in the
    master-slave fashion, and messages are replicated between
    master and slave, so the slave can take over if the master
    has died.
    (http://aidm.googlecode.com/svn/trunk/apache-site/research/papers/mb2.pdf)

    What RabbitMQ clustering doesn't do is provide guarantees against message loss.
    Even if you do everything right (set your messages, queues and exchanges to
    durable, etc.), when a Rabbit cluster node dies, the messages in queues on that
    node can disappear. This is because RabbitMQ doesn't replicate the contents
    of queues throughout the cluster. They live only on the node that owns the
    queue.
    (http://www.cybershovel.com/b/RabbitMQinAction.pdf
    http://pdf.th7.cn/down/files/1312/RabbitMQ%20in%20Action.pdf?yundunkey=1c4e3306d3a07226a40e927b533a8c1841426173782_179979013)

    }
    \item [Zero MQ] \hfill \\
    {}
\end{description}


\subsection{Big data Broker}
\begin{description}
    \item [Apache Kafka] \hfill \\
        { (What it is) (Creator) (License) (Characteristics according features) }
    \item [Amazon Kinesis] \hfill \\
    {}
    \item [Scribe] \hfill \\
    {}
    \item [Kastrell] \hfill \\
    {}
    \item [Apache Flume] \hfill \\
    {}
\end{description}

\subsection{Enterprise Service Bus}


\section{Conslusion}

\todo[inline]{Welches System passt für welchen Anwendungsfall und welche nicht (Gründe)}
