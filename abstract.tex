\chapter{Abstract}

The aim of this thesis is to provide a summary of the current state of
message oriented-middleware and eventually build a message broker in
Haskell, adapted from the concepts of Apache Kafka which was originally
built by LinkedIn. The implementation shall provide basic
functionalities such as producing and consuming messages, with the aim
of reaching equal performance as Apache in a non-clustered setup. The
Apache Kafka Protocol is being used as the underlying wire-protocol and
is implemented in a standalone library on top of which a separate client
library is provided. Thus, the Haskell Message Broker (HMB) as well as
its producer and consumer clients have been successfully proofed as
compatible with Apache Kafka.

This thesis first examines the fundamental concepts behind messaging and
discloses the needs for message brokers.  In a second stage of this
pre-study, the purpose of event-streaming is described containing a
comparison of batch and stream processing which shall explain the
differences in their nature. Finally the concept and features of Apache
Kafka is presented.  Insights into the implementation is provided in the
technical report and is split into two stages.  At first, the protocol
and client library is introduced. Subsequently the broker implementation
is explained including its capabilities as well as the provided
feature-set. After all, the broker is applied to a benchmark against
Apache Kafka. 

The results of this proof of concept show that Haskell is well suited to
build messaging applications as well as implementing protocols based on
context free grammars. The performance which is provided by the Haskell
Message Broker was able to hit the performance of Apache Kafka for some
tests during the benchmark. For the most tested scenarios the
performance suffers as HMB is not sufficiently optimized yet. However,
the Haskell Message Broker is a well established basis of a
state-of-the-art message broker implementation. The authors recommend to
apply further optimization techniques as well as extending the
feature-set before any other use.
