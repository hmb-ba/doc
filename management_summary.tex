\chapter{Management Summary}

\section*{Introduction}

This thesis aims to adapt the concept of Apache Kafka and build a messaging
system, namely Haskell Message Broker (HMB), using the functional programming
language Haskell. The focus, hereby, lies on the implementation of a stable
and scalable server application system as well as on building a log
subsystem, which in Apache Kafka is considered to be the most important
feature. As Kafka comes with its own wire-protocol, a part of this thesis
will focus on a fully compatible implementation of the Apache Kafka Protocol
in Haskell, including a client library allowing Haskell applications to take
use of the protocol implementation and communicate with Apache Kafka or HMB. 

\section*{Approach}

Becoming familiar with the state of the art in messaging, especially event
streaming, was inevitable for us to be able to create our own messaging
system. As an essential part of this prestudy we analysed the approach and
functionality of Apache Kafka. During the first third of our work, we became
familiar with the functional programming paradigm and the programming
language Haskell, which we studied intensively. After the prestudy phase, we
developed an architectural prototype to demonstrate some very basic
functionality of a message broker. With this in our hands, we then continued
working out the details for the protocol implementation and the server
application. A code review by expert Simon Meier helped us to tweak our code
and improve its efficiency. Finally, we tested our system under heavy load in
order to optimize the performance of our application.

\section*{Results}

The first result of this thesis is the prestudy documentation. It provides an
insight into messaging fundamentals and takes a closer look at Apache Kafka
and related topics. It could potentially be used as an academic amendment for
existing lectures. Another result is the implementation of the Kafka protocol
in Haskell. The design decision of separating protocol related code from the
broker implementation led to a stand-alone product, which can be used as a
library for different projects. The open sourced code has already been highly
praised by the Haskell community and found a handful of contributors who
helped uncovering minor issues. The resulting broker application provides a
server with basic networking functionality and persisting messages. It adapts
some features from Apache Kafka and provides the ability to produce and
consume data. It supports Kafka clients as it is based on the protocol
implementation mentioned above. Simple console clients are provided to
demonstrate functionality.
